\documentclass{article}

\usepackage{amsmath}

\usepackage{amsthm}
\newtheorem{theorem}{Theorem}
\newtheorem{definition}{Definition}[section]
\newtheorem{corollary}{Corollary}[theorem]
\newtheorem{lemma}[theorem]{Lemma}
\newtheorem{example}[theorem]{Example}
\newtheorem{proposition}[theorem]{Proposition}


\begin{document}
	Definitions are crucial to understand this paper.

	\begin{definition}{Sequence}:
		E is a set.$N \leq M$.A sequence over E is a discrete map.
		\begin{align*}
			\phi : [N:M] &\rightarrow E\\
			k &\maps \phi(k)
		\end{align*}
	\end{definition}

	\begin{definition}{N-gram}:
		Let $s \in F_N(E)$ (Sequences of length n) and $0 \leq i \leq j < N$. Then 
		\begin{align*}
			\phi_j^i : [i:j] &\rightarrow E\\
			&k\maps \phi(k)
		\end{align*}
	\end{definition}

	\begin{definition}{Melody}
		Let E be an event space. A \texbf{finite} , \texbf{discrete} map
		\begin{align*}
			\mu : [0:N-1] &\rightarrow R x E\\
				n&\maps (t_n , p_n)
		\end{align*}
		is called melody iff \[ t_n < t_m \iff n < m\].
	\end{definition}

	\section*{N-gram measures}
		Three measure being introduced
		\begin{enumerate}
			\item Count-Distinct
			\item Sum-Common
			\item Ukkonen measure
		\end{enumerate}

		\begin{definition}{Frequency of an N-gram r w.r.t a sequence s}
			\[
				f_s(r) \sum_{u\in \underbrace{S_n(s)}_{\text{set of all N-grams of s}}} \\ \underbrace{\delta(u,r)}_{\delta \text{as in Kronecker}}
			\]
		\end{definition}
		The set of all distinct N-grams of a sequence s will be written \textbf{n}(s).


		\begin{definition}{Count-Distinct}
			Let s and t be two sequences over an event space E and let $0 < n \leq min(|s|,|t|)$.

			The CDM is the count of N-grams common to both sequences:

			\[
				S_d(s,t) = \sum_{r\in n(s) \cup n(t)} 1
			\]
		\end{definition}

		\begin{definition}{Sum-Common}
			SCM is the sum of frequencies of N-grams common to both sequences.

			\[
				S_c(s,t) = \sum_{r\in n(s) \cup n(t)} f_s(r) + f_t(r)
			\]

		\end{definition}
			



\end{document}