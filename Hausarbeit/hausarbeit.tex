\documentclass{llncs}
\usepackage{makeidx}

\title{Symbolic Music Similarity}
\author{Ali Bektas \and Paul Kröger}
\date{April 2020}


\usepackage{graphicx}
\graphicspath{{./images/}}
\usepackage{verbatim}
\usepackage{mathtools}





\begin{document}
	
	\frontmatter  

	%Die Arbeit muss ein Inhaltsverzeichnis enthalten. Bitte halten Sie es schlicht. 
	\tableofcontents

	\mainmatter
	%Eine Ausarbeitung hat einen Titel, einen (oder mehrere) Verfasser, einen Betreuer, ein
	%Erstellungsdatum und ist in einem Seminar in einem Jahr verfasst worden. Geben Sie das
	%alles auf einer Titelseite an. 
	\title{Symbolic Melodic Music Similarity}
	\titlerunning{Symbolic Music Similarity}
	\author{Ali Bektas\inst{1} \and Paul Kröger\inst{1}}
	\authorrunning{Bektas \and Kröger} 	% abbreviated author list (for running head)
	\tocauthor{Ali Bektas (Humboldt Universität zu Berlin),
	Paul Kröger(Humboldt Universität zu Berlin)}
	\institute{Humboldt Universität zu Berlin\\}
	\maketitle


	\begin{abstract}
	With an increasing need to classify musical data in terms of their content 
	the question of similarity between musical objects becomes a challange. While 
	there is not a single similarity function to answer all different needs , various approaches
	have been introduced to the field in recent decades. In our paper , we introduce the approaches
	which mainly focus on symbolic data , that is , a representation of the content based on an alphabet
	of symbols and syntax.
	\end{abstract}

	\section{Introduction}
        Since about 10th century 

	\section{}
	
	\section{Evaluation}
	
	\section{Conclusion}

	
\end{document}
